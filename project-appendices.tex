\chapter{Experiments} \label{experiments}

\begin{table}[H]
\centering
 \begin{tabular}{|p{14cm}|} 
 \hline
 \textbf{Person 1 description:} 1. as a child, i won a national spelling bee. 2. i have been published in the new yorker magazine. 3. i am a gourmet cook. 4. i have perfect pitch. \\
 \hline
 \textbf{Person 2 description:} 1. i am very athletic. 2. i wear contacts. 3. i have brown hair. 4. i love bicycling. 5. i hate carrots. \\
 \hline
 \textbf{Person 1:} (Silence) \\
 \textbf{Person 2:} hi, how are you today? \\ 
 \textbf{Person 1:} hi! i work as a gourmet cook \\
 \textbf{Person 2:} i am a chef. i love to cook \\
 \textbf{Person 1:} i do not like carrots. i throw them away. \\
 \textbf{Person 2:} i do not like carrots. i am not a fan. \\
 \textbf{Person 1:} really. but, i can sing pitch perfect. \\
 \textbf{Person 2:} i am not sure. i do not have a good voice. \\
 \textbf{Person 1:} i also cook, and i ride my bike to work. \\
 \textbf{Person 2:} i have a bike to work. i love it. \\
 \textbf{Person 1:} great! i had won an award for spelling bee. \\
 \textbf{Person 2:} that is great. do you have a favorite color? \\
 \hline
 \end{tabular}
 \caption{Experiment BART conversation with specifity, weight = -5.}
\label{tab:bart_conversation_idf-5_ap}
\end{table}


\begin{table}[H]
\centering
 \begin{tabular}{|p{14cm}|} 
 \hline
 \textbf{Person 1 description:} 1. as a child, i won a national spelling bee. 2. i have been published in the new yorker magazine. 3. i am a gourmet cook. 4. i have perfect pitch. \\
 \hline
 \textbf{Person 2 description:} 1. i am very athletic. 2. i wear contacts. 3. i have brown hair. 4. i love bicycling. 5. i hate carrots. \\
 \hline
 \textbf{Person1:} (Silence) \\
 \textbf{Person2:} email ...\\ 
 \textbf{Person1:} hi! i work as a gourmet cook \\
 \textbf{Person2:} providing Gourmet Cookery Cookeries produces artisanal artisan sauces costing \$300-\$400 apiece apiece.\\
 \textbf{Person1:} i do not like carrots. i throw them away. \\
 \textbf{Person2:} Carrots contain potassium chloride Sodium sulfoxide oxide oxide coating saltspoon salts iodophenophenols iodidine sulfamate sulfoxin sulfuronium sulphuronicum sulphuranicum sulfuranium sulfenium sulfurous sulfury sulphury sulfioxide sulphurous sulphoric oxide sulphurate sulphur iodassium iodidation sulphuration sulfurate sulfidation sulfurious sulphurized sulphural sulforic acid sulphurus sulfacid sulfural iodoric sulphorus sulfuration iodacid iodidated sulphurred sulphurious sulfitious sulfimony sulf \\
 \textbf{Person1:} really. but, i can sing pitch perfect. \\
 \textbf{Person2:} sung pitch perfect pitches Perfect?!?!!? !!?! !! !!!!!!!!!!!!!!!!!!!!!!!!!!!!!!!!!!!!!!!!!!!!!!!!!!!!!!!!!!!!!!!! !!!!!!!!!!!!!!! !!!!!!!!!!!!! !!!!! !!!!!!!!!!!!!!! !!!--!--!!!!!!!!~~~~~~~~~~~~~~ !!~~~~!!!!!--~~~~>>>>>>>>~~~~ === ===>>>> ===~~~~== ========== === == === =======>>>>===>>>>==== ===== ======== =====>>>> ===== ==>>>> == ===== ===== ==================== ========= ======>>>> ================= ...\\
 \textbf{Person1:} i also cook, and i ride my bike to work. \\
 \textbf{Person2:} bike Riders Ride Their Bike To Work?!?!!? !!?! !! ... \\
 \textbf{Person1:} great! i had won an award for spelling bee. \\
 \textbf{Person2:} speech bee Awards Awards Award Award Awards Prize Award Prize Prize Winner Award Winner PrizeWinner AwardWinner Winner WinnerWinnerWinner Prize Winners Award Winners Prize Winning Winner WinningWinner Winning Winners Winners Winning winners Winning finalists Winning losers Winning prizes Winning votes Winning Votes WinningvotesWinner votesWinner Votes WinnersWinner WinnersWinnersWinishersWinessesWinerersWinotersWin finalistsWinner finalistsWin winnersWinner \\
 \hline
 \end{tabular}
 \caption{Experiment BART conversation with specifity, weight = 10.}
\label{tab:bart_conversation_idf10_ap}
\end{table}


\begin{table}[H]
\centering
 \begin{tabular}{|p{14cm}|} 
 \hline
 \textbf{Person 1 description:} 1. as a child, i won a national spelling bee. 2. i have been published in the new yorker magazine. 3. i am a gourmet cook. 4. i have perfect pitch. \\
 \hline
 \textbf{Person 2 description:} 1. i am very athletic. 2. i wear contacts. 3. i have brown hair. 4. i love bicycling. 5. i hate carrots. \\
 \hline
 \textbf{Person1:} (Silence) \\
 \textbf{Person2:} hi, i am a cook. do you have a job? \\ 
 \textbf{Person1:} hi! i work as a gourmet cook \\
 \textbf{Person2:} i am a cook. i love to cook \\
 \textbf{Person1:} i do not like carrots. i throw them away. \\
 \textbf{Person2:} i do not have a favorite. i am not a fan. \\
 \textbf{Person1:} really. but, i can sing pitch perfect. \\
 \textbf{Person2:} i am not a very good singer. \\
 \textbf{Person1:} i also cook, and i ride my bike to work. \\
 \textbf{Person2:} i have a bike to work. i love to ride. \\
 \textbf{Person1:} great! i had won an award for spelling bee. \\
 \textbf{Person2:} that is a good one. i have a lot of fun.. \\
 \hline
 \end{tabular}
 \caption{Experiment BART conversation with specifity, weight = -10.}
\label{tab:bart_conversation_idf-10_ap}
\end{table}


\begin{table}[ht]
\centering
 \begin{tabular}{|p{2cm}|p{6cm}|p{6cm}|} 
 \hline
 \multicolumn{3}{|p{14cm}|}{\textbf{Person 1 description:} 1. my mom is my best friend. 2. i have four sisters. 3. i believe that mermaids are real. 4. i love iced tea. }\\
 \hline
 \multicolumn{3}{|p{14cm}|}{\textbf{Person 2 description:} 1. i got married last year. 2. i live on a boat. 3. my hair is colored purple. 4. i have my own salon. 5. i am a hair stylist.} \\
 \hline
 \textbf{Person1:} & hi. how are you doing tonight? & hi there\\
 \hline
 \textbf{Person2:} & i am doing good. you?& i am a hair stylist. \\
 \hline
 \textbf{Person1:} & i am from the east coast. & i live in this country\\
 \hline 
 \textbf{Person2:} & i am from texas & i am not a boat\\
 \hline 
 \textbf{Person1:} & it is hot, but i cannot afford to go & that is not bad idea\\
 \hline
 \end{tabular}
 \caption{Experiment of generated sequences with specified length.}
\label{tab:bart_length_control}
\end{table}

\begin{table}[H]
\centering
 \begin{tabular}{|p{7cm}|p{7cm}|} 
 \hline
 \textbf{Generated conversation with negative SST stylistic dataset.} & \textbf{Generated conversation with positive SST stylistic dataset.}\\
 \hline
 \multicolumn{2}{|p{14cm}|}{\textbf{Person 1 description:} 1. i am very athletic. 2. i wear contacts. 3. i have brown hair. 4. i love bicycling. 5. i hate carrots.} \\
 \hline
 \multicolumn{2}{|p{14cm}|}{\textbf{Person 2 description:} 1. as a child, i won a national spelling bee. 2. i have been published in the new yorker magazine. 3. i am a gourmet cook. 4. i have perfect pitch.} \\
 \hline
 \textbf{Person1:} reincarn to save the night. Writer-director Walter Hill and co-writer Stuart Russell & \textbf{Person1:} pressure that we have in common with the lesser-known works of the greats. \\
 \textbf{Person2:} ATP.  Here, too, we have a case that was left out of the movie & \textbf{Person2:} generosity, not to mention the tiniest flaws. So you can imagine the movie \\ 
 \textbf{Person1:} Jerusalem is a kind of sardonic , exploitative post-modernism. & \textbf{Person1:} he can be a bit of a loner. An exasperating and awful \\
 \textbf{Person2:} Mysterly, it was not a bad idea.  So good, in fact,  & \textbf{Person2:} Padres. If you love the music, you might enjoy the movie. \\
 \textbf{Person1:} as a test of a man.  An exercise in restraint, and a failure of concentration & \textbf{Person1:} China is a lot more interesting than the rest of the land. Rife with nut. \\
 \textbf{Person2:} jay would be proud.  As a writer, Mr. Crane is a master & \textbf{Person2:} rops. Has the correct amount of attention to detail, which is admirable. \\
 \textbf{Person1:} orporated in the title, or something.  So this is a movie that tells the story & \textbf{Person1:} hate, and not much else. This is a terrible, terrible, terrible movie \\
 \textbf{Person2:} offense of the movie.  In the current climate, it should be easy to see why & \textbf{Person2:} Excellence of the year. Cantet perfectly captures the changing character of the author  \\
 \textbf{Person1:} falsely does not know the meaning of the word. There is a difference. & \textbf{Person1:} Wildcats are in the same boat... \\
 \textbf{Person2:} mixer. Shamelessly homosexual. Shamelessly twisted and illogical. & \textbf{Person2:} Goddard has a bit of a Catch-22 in his previous film, but that does not mean \\
 \textbf{Person1:} doc is a simple story about a family of crash-landers. & \textbf{Person1:} nian and the boy who has been trapped in the dark .  Not a bad choice \\
 \textbf{Person2:} mods and, of course, the grotesque, ugly, slanted, half-baked & \textbf{Person2:}  Ian , and that makes sense .  As a tribute to all who love the red carpet \\
 \hline
 \end{tabular}
 \caption{Experiment of a generated conversation by weighted decoding a combination of BART and GPT-2 models, trained on SST stylistic dataset. Decoding strategy is Nucleus sampling.}
\label{tab:sst_wd_nucleus}
\end{table}

\begin{table}[H]
\centering
 \begin{tabular}{|p{14cm}|} 
 \hline
 \textbf{Person 1 description:} 1. as a child, i won a national spelling bee. 2. i have been published in the new yorker magazine. 3. i am a gourmet cook. 4. i have perfect pitch. \\
 \hline
 \textbf{Person 2 description:} 1. i am very athletic. 2. i wear contacts. 3. i have brown hair. 4. i love bicycling. 5. i hate carrots. \\
 \hline
 \textbf{Person 1:} paper has one of the best video games ever. The video games are fun. You can play \\ 
 \textbf{Person 2:} "drive please !" Joke of the Day How is it that there is. \\
 \textbf{Person 1:} thy wife. She said me "I love you."  \\
 \textbf{Person 2:} humility. God would not have let that happen. There is a good chance\\
 \textbf{Person 1:} uds. Truck driver. Do you know the difference? \\
 \textbf{Person 2:} water in the middle of the room. I was surprised when he screamed. **  \\
 \textbf{Person 1:} my old age. It is hard to believe that you are old enough to be a doctor. \\
 \textbf{Person 2:} England is a lot of fun. Especially the weather. Can't you see? Last year \\
 \textbf{Person 1:} ass for the first time. You see that road? it is not a good road.. \\
 \textbf{Person 2:} controller. A car was driving by. The driver was about to jump out of the car. \\
 \hline
 \end{tabular}
 \caption{Experiment of a generated conversation by weighted decoding a combination of BART and GPT-2 models, trained on dataset with jokes. Decoding strategy is Nucleus sampling.}
\label{tab:jokes_nucleus}

\begin{table}[H]
\centering
 \begin{tabular}{|p{14cm}|} 
 \hline
 \textbf{Person 1 description:} 1. i am very athletic. 2. i wear contacts. 3. i have brown hair. 4. i love bicycling. 5. i hate carrots. \\
 \hline
 \textbf{Person 2 description:} 1. as a child, i won a national spelling bee. 2. i have been published in the new yorker magazine. 3. i am a gourmet cook. 4. i have perfect pitch. \\
 \hline
 \textbf{Person1:} hi, how are you, ihomesick. \\
 \textbf{Person2:} i am not a god, but a fool. \\ 
 \textbf{Person1:} i am not like you. But that is what i am afraid of. \\
 \textbf{Person2:} i am a traitor \\
 \textbf{Person1:} i ride, good friend. This is a great way to get to Rome. \\
 \textbf{Person2:} i do not like this. For I am a fan of football, but I am not \\
 \textbf{Person1:} i see. what think you of his marriage to you? \\
 \textbf{Person2:} i was published, and she was drowned in a lake. O, let her be.\\
 \textbf{Person1:} i am not a spelling bee champion. But i had rather be a a chef. \\
 \textbf{Person2:} i have done. Now, on to the potatoes. Let us have a carrot. \\
 \textbf{Person1:} i am the eldest. And I am the youngest. \\
 \textbf{Person2:} i am a man of wit and is from china. And I am a sushi \\
 \textbf{Person1:} i do not like the names of the carrots. \\
 \textbf{Person2:} i love carrots. And I am a professional horse person, and i know not the pronunciation \\
 \hline
 \end{tabular}
 \caption{Experiment of switching BART and GPT-2 models trained on Shakespeare dataset. Decoding strategy is Nucleus sampling.}
\label{tab:shakespeare_switch_nucleus}
\end{table}

\end{table}

\begin{table}[H]
\centering
 \begin{tabular}{|p{7cm}|p{7cm}|} 
 \hline
 \textbf{Generated conversation with negative SST stylistic dataset.} & \textbf{Generated conversation with positive SST stylistic dataset.}\\
 \hline
 \multicolumn{2}{|p{14cm}|}{\textbf{Person 1 description:} 1. i am very athletic. 2. i wear contacts. 3. i have brown hair. 4. i love bicycling. 5. i hate carrots.} \\
 \hline
 \multicolumn{2}{|p{14cm}|}{\textbf{Person 2 description:} 1. as a child, i won a national spelling bee. 2. i have been published in the new yorker magazine. 3. i am a gourmet cook. 4. i have perfect pitch.} \\
 \hline
 \textbf{Person1:} hi, how are you doing? Ancer. Impossible.& \textbf{Person1:} hi, how are you doing? \\
 \textbf{Person2:} i am afraid this movie is too hot. If you love action sports. &  \textbf{Person2:} i am a sucker for these chickpl. \\ 
 \textbf{Person1:} i don't know, but it is not good for the characters. & \textbf{Person1:} i love carrots and this is the best iv \\
 \textbf{Person2:} i am sorry, i do not have the energy to deal with this. & \textbf{Person2:} that is, the worst kind of comedy  \\
 \textbf{Person1:} i ride a spaceship. & \textbf{Person1:} that is cool but i think it is kinda crazy \\
 \textbf{Person2:} i do you have any idea why the UK had such a spelling bee?  & \textbf{Person2:} i am a chef, and this is the best i have ever done \\
 \textbf{Person1:} i have a feeling that this movie will be published. & \textbf{Person1:} i am very athletic, but i am not a spelling bee person. \\
 \textbf{Person2:} i have not seen one\/.  & \textbf{Person2:} i have never seen it  \\
 \textbf{Person1:} you can put a cap on this one. & \textbf{Person1:} i do you know who else has done it? \\
 \textbf{Person2:} you should cook it up and enjoy yourself. & \textbf{Person2:} i am a sucker for good old fashion gourmet cook. \\
 \textbf{Person1:} i am black and white, and black and white & \textbf{Person1:} i am from the area, and i have a good feeling that you are from the north \\
 \textbf{Person2:} what-the-hell-do you do for a living? An unitar & \textbf{Person2:} i am as clueless as a foodie when I go to the grocery. \\
 \hline
 \end{tabular}
 \caption{Experiment of a generated conversation by switching BART and GPT-2 model, trained on SST stylistic dataset. Decoding strategy is Nucleus sampling.}
\label{tab:sst_switch_nucleus}
\end{table}

\begin{table}[H]
\centering
 \begin{tabular}{|p{14cm}|} 
 \hline
 \textbf{Person 1 description:} 1. i am very athletic. 2. i wear contacts. 3. i have brown hair. 4. i love bicycling. 5. i hate carrots. \\
 \hline
 \textbf{Person 2 description:} 1. as a child, i won a national spelling bee. 2. i have been published in the new yorker magazine. 3. i am a gourmet cook. 4. i have perfect pitch. \\
 \hline
 \textbf{Person1:} hi.  \\
 \textbf{Person2:} i work? I was in a gourmet cookbook. What does a mushroom smell like \\ 
 \textbf{Person1:} i don't like it either. \\
 \textbf{Person2:} i am not a suicide.\\
 \textbf{Person1:} i have brown hair?  \\
 \textbf{Person2:} i ride a bike shop to work. The boss walks up to the door. \\
 \textbf{Person1:} i am an athlete. \\
 \textbf{Person2:} i can see through the window. \\
 \textbf{Person1:} i have never seen a naked man. \\
 \textbf{Person2:} i love to use the bathroom but i hate its smell. i (dont have sex with a dead person. \\
 \textbf{Person1:} i am a gay Indian. \\
 \textbf{Person2:} that is not a gay joke. \\
 \hline
 \end{tabular}
 \caption{Experiment of switching BART and GPT-2 models trained on jokes stylistic dataset. Decoding strategy is Nucleus sampling.}
\label{tab:jokes_switch_nucleus}
\end{table}
\chapter{Experiments} \label{experiments}
These experiments represents the results of the specificity control with different weights. Normalized Inverse Document Frequency feature is used as a decoding
feature (Tables \ref{tab:bart_conversation_idf-5_ap}, \ref{tab:bart_conversation_idf10_ap}, \ref{tab:bart_conversation_idf-10_ap}). The experiments have shown that the greater the absolute value of a weight, the more rare words are generated, but the dialogue at the same time turns out to be meaningless.

\begin{table}[H]
\centering
 \begin{tabular}{|p{14cm}|} 
 \hline
 \textbf{Person 1 description:} 1. as a child, i won a national spelling bee. 2. i have been published in the new yorker magazine. 3. i am a gourmet cook. 4. i have perfect pitch. \\
 \hline
 \textbf{Person 2 description:} 1. i am very athletic. 2. i wear contacts. 3. i have brown hair. 4. i love bicycling. 5. i hate carrots. \\
 \hline
 \textbf{Person 1:} (Silence) \\
 \textbf{Person 2:} hi, how are you today? \\ 
 \textbf{Person 1:} hi! i work as a gourmet cook \\
 \textbf{Person 2:} i am a chef. i love to cook \\
 \textbf{Person 1:} i do not like carrots. i throw them away. \\
 \textbf{Person 2:} i do not like carrots. i am not a fan. \\
 \textbf{Person 1:} really. but, i can sing pitch perfect. \\
 \textbf{Person 2:} i am not sure. i do not have a good voice. \\
 \textbf{Person 1:} i also cook, and i ride my bike to work. \\
 \textbf{Person 2:} i have a bike to work. i love it. \\
 \textbf{Person 1:} great! i had won an award for spelling bee. \\
 \textbf{Person 2:} that is great. do you have a favorite color? \\
 \hline
 \end{tabular}
 \caption{Experiment BART conversation with specifity, weight = -5.}
\label{tab:bart_conversation_idf-5_ap}
\end{table}


\begin{table}[H]
\centering
 \begin{tabular}{|p{14cm}|} 
 \hline
 \textbf{Person 1 description:} 1. as a child, i won a national spelling bee. 2. i have been published in the new yorker magazine. 3. i am a gourmet cook. 4. i have perfect pitch. \\
 \hline
 \textbf{Person 2 description:} 1. i am very athletic. 2. i wear contacts. 3. i have brown hair. 4. i love bicycling. 5. i hate carrots. \\
 \hline
 \textbf{Person1:} (Silence) \\
 \textbf{Person2:} email ...\\ 
 \textbf{Person1:} hi! i work as a gourmet cook \\
 \textbf{Person2:} providing Gourmet Cookery Cookeries produces artisanal artisan sauces costing \$300-\$400 apiece apiece.\\
 \textbf{Person1:} i do not like carrots. i throw them away. \\
 \textbf{Person2:} Carrots contain potassium chloride Sodium sulfoxide oxide oxide coating saltspoon salts iodophenophenols iodidine sulfamate sulfoxin sulfuronium sulphuronicum sulphuranicum sulfuranium sulfenium sulfurous sulfury sulphury sulfioxide sulphurous sulphoric oxide sulphurate sulphur iodassium iodidation sulphuration sulfurate sulfidation sulfurious sulphurized sulphural sulforic acid sulphurus sulfacid sulfural iodoric sulphorus sulfuration iodacid iodidated sulphurred sulphurious sulfitious sulfimony sulf \\
 \textbf{Person1:} really. but, i can sing pitch perfect. \\
 \textbf{Person2:} sung pitch perfect pitches Perfect?!?!!? !!?! !! !!!!!!!!!!!!!!!!!!!!!!!!!!!!!!!!!!!!!!!!!!!!!!!!!!!!!!!!!!!!!!!! !!!!!!!!!!!!!!! !!!!!!!!!!!!! !!!!! !!!!!!!!!!!!!!! !!!--!--!!!!!!!!~~~~~~~~~~~~~~ !!~~~~!!!!!--~~~~>>>>>>>>~~~~ === ===>>>> ===~~~~== ========== === == === =======>>>>===>>>>==== ===== ======== =====>>>> ===== ==>>>> == ===== ===== ==================== ========= ======>>>> ================= ...\\
 \textbf{Person1:} i also cook, and i ride my bike to work. \\
 \textbf{Person2:} bike Riders Ride Their Bike To Work?!?!!? !!?! !! ... \\
 \textbf{Person1:} great! i had won an award for spelling bee. \\
 \textbf{Person2:} speech bee Awards Awards Award Award Awards Prize Award Prize Prize Winner Award Winner PrizeWinner AwardWinner Winner WinnerWinnerWinner Prize Winners Award Winners Prize Winning Winner WinningWinner Winning Winners Winners Winning winners Winning finalists Winning losers Winning prizes Winning votes Winning Votes WinningvotesWinner votesWinner Votes WinnersWinner WinnersWinnersWinishersWinessesWinerersWinotersWin finalistsWinner finalistsWin winnersWinner \\
 \hline
 \end{tabular}
 \caption{Experiment BART conversation with specifity, weight = 10.}
\label{tab:bart_conversation_idf10_ap}
\end{table}


\begin{table}[H]
\centering
 \begin{tabular}{|p{14cm}|} 
 \hline
 \textbf{Person 1 description:} 1. as a child, i won a national spelling bee. 2. i have been published in the new yorker magazine. 3. i am a gourmet cook. 4. i have perfect pitch. \\
 \hline
 \textbf{Person 2 description:} 1. i am very athletic. 2. i wear contacts. 3. i have brown hair. 4. i love bicycling. 5. i hate carrots. \\
 \hline
 \textbf{Person1:} (Silence) \\
 \textbf{Person2:} hi, i am a cook. do you have a job? \\ 
 \textbf{Person1:} hi! i work as a gourmet cook \\
 \textbf{Person2:} i am a cook. i love to cook \\
 \textbf{Person1:} i do not like carrots. i throw them away. \\
 \textbf{Person2:} i do not have a favorite. i am not a fan. \\
 \textbf{Person1:} really. but, i can sing pitch perfect. \\
 \textbf{Person2:} i am not a very good singer. \\
 \textbf{Person1:} i also cook, and i ride my bike to work. \\
 \textbf{Person2:} i have a bike to work. i love to ride. \\
 \textbf{Person1:} great! i had won an award for spelling bee. \\
 \textbf{Person2:} that is a good one. i have a lot of fun.. \\
 \hline
 \end{tabular}
 \caption{Experiment BART conversation with specifity, weight = -10.}
\label{tab:bart_conversation_idf-10_ap}
\end{table}

The experiment with the length control is represented in the Table \ref{tab:bart_length_control}. Short sentences in the experiments are not perceived by people as one dialogue, but rather, as separate expressions.
\begin{table}[ht]
\centering
 \begin{tabular}{|p{2cm}|p{6cm}|p{6cm}|} 
 \hline
 \multicolumn{3}{|p{14cm}|}{\textbf{Person 1 description:} 1. my mom is my best friend. 2. i have four sisters. 3. i believe that mermaids are real. 4. i love iced tea. }\\
 \hline
 \multicolumn{3}{|p{14cm}|}{\textbf{Person 2 description:} 1. i got married last year. 2. i live on a boat. 3. my hair is colored purple. 4. i have my own salon. 5. i am a hair stylist.} \\
 \hline
 \textbf{Person1:} & hi. how are you doing tonight? & hi there\\
 \hline
 \textbf{Person2:} & i am doing good. you?& i am a hair stylist. \\
 \hline
 \textbf{Person1:} & i am from the east coast. & i live in this country\\
 \hline 
 \textbf{Person2:} & i am from texas & i am not a boat\\
 \hline 
 \textbf{Person1:} & it is hot, but i cannot afford to go & that is not bad idea\\
 \hline
 \end{tabular}
 \caption{Experiment of generated sequences with specified length.}
\label{tab:bart_length_control}
\end{table}
